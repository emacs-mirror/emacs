% Reference Card for GNU Emacs

% Copyright (C) 2025--2026 Free Software Foundation, Inc.

% Author: Stephen Gildea <stepheng+emacs@gildea.com>
% Spanish translation: Elias Gabriel Perez <eg642616@gmail.com>

% This document is free software: you can redistribute it and/or modify
% it under the terms of the GNU General Public License as published by
% the Free Software Foundation, either version 3 of the License, or
% (at your option) any later version.

% As a special additional permission, you may distribute reference cards
% printed, or formatted for printing, with the notice "Released under
% the terms of the GNU General Public License version 3 or later"
% instead of the usual distributed-under-the-GNU-GPL notice, and without
% a copy of the GPL itself.

% This document is distributed in the hope that it will be useful,
% but WITHOUT ANY WARRANTY; without even the implied warranty of
% MERCHANTABILITY or FITNESS FOR A PARTICULAR PURPOSE.  See the
% GNU General Public License for more details.

% You should have received a copy of the GNU General Public License
% along with GNU Emacs.  If not, see <https://www.gnu.org/licenses/>.


% This file is intended to be processed by plain TeX (TeX82).
%
% The final reference card has six columns, three on each side.
% This file can be used to produce it in any of three ways:
% 1 column per page
%    produces six separate pages, each of which needs to be reduced to 80%.
%    This gives the best resolution.
% 2 columns per page
%    produces three already-reduced pages.
%    You will still need to cut and paste.
% 3 columns per page
%    produces two pages which must be printed sideways to make a
%    ready-to-use 8.5 x 11 inch reference card.
%    For this you need a dvi device driver that can print sideways.
% Which mode to use is controlled by setting \columnsperpage.
%
% To compile and print this document:
% tex refcard.tex
% dvips -t landscape refcard.dvi

% Thanks to Cecilio Pardo, Mauro Aranda and Stephen Berman for
% the suggestions.

%**start of header
\newcount\columnsperpage
\newcount\letterpaper

% This file can be printed with 1, 2, or 3 columns per page.
% Specify how many you want here.
\columnsperpage=3

% Set letterpaper to 0 for A4 paper, 1 for letter (US) paper.  Useful
% only when columnsperpage is 2 or 3.
\letterpaper=1

% PDF output layout.  0 for A4, 1 for letter (US), a `l' is added for
% a landscape layout.
\input pdflayout.sty
\pdflayout=(1l)

% Nothing else needs to be changed below this line.

\input emacsver.tex

\def\shortcopyrightnotice{\vskip 1ex plus 2 fill
  \centerline{\small \copyright\ \year\ Free Software Foundation, Inc.
  Permissions on back.}}

\def\copyrightnotice{
\vskip 1ex plus 2 fill\begingroup\small
\centerline{Copyright \copyright\ \year\ Free Software Foundation, Inc.}
\centerline{For GNU Emacs version \versionemacs}
\centerline{Designed by Stephen Gildea}
\centerline{Translated by Elias Gabriel P{\'e}rez}

Released under the terms of the GNU General Public License version 3 or later.

For more Emacs documentation, and the \TeX{} source for this card,
see the Emacs distribution, or {\tt https://www.gnu.org/software/emacs}
\endgroup}

% make \bye not \outer so that the \def\bye in the \else clause below
% can be scanned without complaint.
\def\bye{\par\vfill\supereject\end}

\newdimen\intercolumnskip	%horizontal space between columns
\newbox\columna			%boxes to hold columns already built
\newbox\columnb

\def\ncolumns{\the\columnsperpage}

\message{[\ncolumns\space
  column\if 1\ncolumns\else s\fi\space per page]}

\def\scaledmag#1{ scaled \magstep #1}

% This multi-way format was designed by Stephen Gildea October 1986.
% Note that the 1-column format is fontfamily-independent.
\if 1\ncolumns			%one-column format uses normal size
  \hsize 4in
  \vsize 10in
  \voffset -.7in
  \font\titlefont=\fontname\tenbf \scaledmag3
  \font\headingfont=\fontname\tenbf \scaledmag2
  \font\smallfont=\fontname\sevenrm
  \font\smallsy=\fontname\sevensy

  \footline{\hss\folio}
  \def\makefootline{\baselineskip10pt\hsize6.5in\line{\the\footline}}
\else				%2 or 3 columns uses prereduced size
  \hsize 3.2in
  \if 1\the\letterpaper
     \vsize 7.95in
  \else
     \vsize 7.65in
  \fi
  \hoffset -.75in
  \voffset -.745in
  \font\titlefont=cmbx10 \scaledmag2
  \font\headingfont=cmbx10 \scaledmag1
  \font\smallfont=cmr6
  \font\smallsy=cmsy6
  \font\eightrm=cmr8
  \font\eightbf=cmbx8
  \font\eightit=cmti8
  \font\eighttt=cmtt8
  \font\eightmi=cmmi8
  \font\eightsy=cmsy8
  \textfont0=\eightrm
  \textfont1=\eightmi
  \textfont2=\eightsy
  \def\rm{\eightrm}
  \def\bf{\eightbf}
  \def\it{\eightit}
  \def\tt{\eighttt}
  \if 1\the\letterpaper
     \normalbaselineskip=.8\normalbaselineskip
  \else
     \normalbaselineskip=.7\normalbaselineskip
  \fi
  \normallineskip=.8\normallineskip
  \normallineskiplimit=.8\normallineskiplimit
  \normalbaselines\rm		%make definitions take effect

  \if 2\ncolumns
    \let\maxcolumn=b
    \footline{\hss\rm\folio\hss}
    \def\makefootline{\vskip 2in \hsize=6.86in\line{\the\footline}}
  \else \if 3\ncolumns
    \let\maxcolumn=c
    \nopagenumbers
  \else
    \errhelp{You must set \columnsperpage equal to 1, 2, or 3.}
    \errmessage{Illegal number of columns per page}
  \fi\fi

  \intercolumnskip=.46in
  \def\abc{a}
  \output={%			%see The TeXbook page 257
      % This next line is useful when designing the layout.
      %\immediate\write16{Column \folio\abc\space starts with \firstmark}
      \if \maxcolumn\abc \multicolumnformat \global\def\abc{a}
      \else\if a\abc
	\global\setbox\columna\columnbox \global\def\abc{b}
        %% in case we never use \columnb (two-column mode)
        \global\setbox\columnb\hbox to -\intercolumnskip{}
      \else
	\global\setbox\columnb\columnbox \global\def\abc{c}\fi\fi}
  \def\multicolumnformat{\shipout\vbox{\makeheadline
      \hbox{\box\columna\hskip\intercolumnskip
        \box\columnb\hskip\intercolumnskip\columnbox}
      \makefootline}\advancepageno}
  \def\columnbox{\leftline{\pagebody}}

  \def\bye{\par\vfill\supereject
    \if a\abc \else\null\vfill\eject\fi
    \if a\abc \else\null\vfill\eject\fi
    \end}
\fi

% we won't be using math mode much, so redefine some of the characters
% we might want to talk about
\catcode`\^=12
\catcode`\_=12

\chardef\\=`\\
\chardef\{=`\{
\chardef\}=`\}

\hyphenation{mini-buf-fer}

\parindent 0pt
\parskip 1ex plus .5ex minus .5ex

\def\small{\smallfont\textfont2=\smallsy\baselineskip=.8\baselineskip}

% newcolumn - force a new column.  Use sparingly, probably only for
% the first column of a page, which should have a title anyway.
\outer\def\newcolumn{\vfill\eject}

% title - page title.  Argument is title text.
\outer\def\title#1{{\titlefont\centerline{#1}}\vskip 1ex plus .5ex}

% section - new major section.  Argument is section name.
\outer\def\section#1{\par\filbreak
  \vskip 3ex plus 2ex minus 2ex {\headingfont #1}\mark{#1}%
  \vskip 2ex plus 1ex minus 1.5ex}

\newdimen\keyindent

% beginindentedkeys...endindentedkeys - key definitions will be
% indented, but running text, typically used as headings to group
% definitions, will not.
\def\beginindentedkeys{\keyindent=1em}
\def\endindentedkeys{\keyindent=0em}
\endindentedkeys

% paralign - begin paragraph containing an alignment.
% If an \halign is entered while in vertical mode, a parskip is never
% inserted.  Using \paralign instead of \halign solves this problem.
\def\paralign{\vskip\parskip\halign}

% \<...> - surrounds a variable name in a code example
\def\<#1>{{\it #1\/}}

% kbd - argument is characters typed literally.  Like the Texinfo command.
\def\kbd#1{{\tt#1}\null}	%\null so not an abbrev even if period follows

% beginexample...endexample - surrounds literal text, such a code example.
% typeset in a typewriter font with line breaks preserved
\def\beginexample{\par\leavevmode\begingroup
  \obeylines\obeyspaces\parskip0pt\tt}
{\obeyspaces\global\let =\ }
\def\endexample{\endgroup}

% key - definition of a key.
% \key{description of key}{key-name}
% prints the description left-justified, and the key-name in a \kbd
% form near the right margin.
\def\key#1#2{\leavevmode\hbox to \hsize{\vtop
  {\hsize=.75\hsize\rightskip=1em
  \hskip\keyindent\relax#1}\kbd{#2}\hfil}}

\newbox\metaxbox
\setbox\metaxbox\hbox{\kbd{M-x }}
\newdimen\metaxwidth
\metaxwidth=\wd\metaxbox

% metax - definition of a M-x command.
% \metax{description of command}{M-x command-name}
% Tries to justify the beginning of the command name at the same place
% as \key starts the key name.  (The "M-x " sticks out to the left.)
\def\metax#1#2{\leavevmode\hbox to \hsize{\hbox to .75\hsize
  {\hskip\keyindent\relax#1\hfil}%
  \hskip -\metaxwidth minus 1fil
  \kbd{#2}\hfil}}

% threecol - like "key" but with two key names.
% for example, one for doing the action backward, and one for forward.
\def\threecol#1#2#3{\hskip\keyindent\relax#1\hfil&\kbd{#2}\hfil\quad
  &\kbd{#3}\hfil\quad\cr}

%**end of header


\title{GNU Emacs: Tarjeta de referencia}

\centerline{(para la versi{\'o}n: \versionemacs)}

\section{Notaci{\'o}n de combinaciones de teclas}

Las notaciones de combinaciones de teclas en Emacs, \kbd{C-x}
es \kbd{Ctrl+X}; \kbd{M-x} es usualmente \kbd{Alt+X}; \kbd{S-x} es
\kbd{Shift+X}; y \kbd{C-M-x} es \kbd{Ctrl+Alt+X}, etc.

\section{Saliendo de Emacs}

\key{minimizar (o suspender en la terminal)}{C-z}
\key{cerrar Emacs}{C-x C-c}

\section{Archivos}

\key{{\bf abrir} un archivo en Emacs}{C-x C-f}
\key{{\bf guardar} un archivo (en el disco)}{C-x C-s}
\key{guardar {\bf todos} los archivos}{C-x s}
\key{{\bf insertar} contenidos de otro archivo en este buffer}{C-x i}
\key{remplazar archivo por otro}{C-x C-v}
\key{guardar buffer a un archivo en especifico}{C-x C-w}
\key{alternar el estado de solo-lectura del buffer}{C-x C-q}

\section{Obtener Ayuda}

El sistema de ayuda es simple. Teclee \kbd{C-h} (o \kbd{F1}) y siga
las opciones. Si est{\'a} usted empezando, teclee \kbd{C-h t} para ir al
{\bf tutorial}.

\key{quitar ventana de ayuda}{C-x 1}
\key{desplazarse por la ventana de ayuda}{C-M-v}

\key{apropos: mostrar comandos que coinciden con una cadena caracteres (string)}{C-h a}
\key{describir la funci{\'o}n asociada a una tecla}{C-h k}
\key{describir una funci{\'o}n}{C-h f}
\key{obtener informaci{\'o}n espec{\'i}fica del modo}{C-h m}

\section{Recuperaci{\'o}n de errores}

\key{{\bf abortar} acci{\'o}n o un comando ejecut{\'a}ndose}{C-g}
\metax{{\bf recuperar} archivos perdidos}{M-x recover-session}
\key{por cierre inesperado}{}
\metax{{\bf deshacer} un cambio no deseado}{C-x u, C-_ {\rm o} C-/}
\metax{restaurar buffer a su contenido actual}{M-x revert-buffer}
\key{recentrar pantalla}{C-l}

\section{B{\'u}squeda incremental}

\key{buscar hacia adelante}{C-s}
\key{buscar hacia atr{\'a}s}{C-r}
\key{buscar una expresi{\'o}n regular}{C-M-s}
\key{buscar inversamente una expresi{\'o}n regular}{C-M-r}

\key{seleccionar anterior cadena de caracteres de la b{\'u}squeda}{M-p}
\key{seleccionar siguiente cadena de caracteres de la b{\'u}squeda}{M-n}
\key{salir de la b{\'u}squeda incremental}{RET}
\key{deshacer efecto del {\'u}ltimo car{\'a}cter}{DEL}
\key{abortar b{\'u}squeda actual}{C-g}

Usa \kbd{C-s} o \kbd{C-r} de nuevo para repetir la b{\'u}squeda en cualquier posici{\'o}n.
Si Emacs sigue buscando, \kbd{C-g} cancela solo la parte no coincidente.

\shortcopyrightnotice

\newcolumn
\section{Movimiento}

\paralign to \hsize{#\tabskip=10pt plus 1 fil&#\tabskip=0pt&#\cr
\threecol{{\bf entidad a trasladar}}{{\bf atr{\'a}s}}{{\bf adelante}}
\threecol{car{\'a}cter}{C-b}{C-f}
\threecol{palabra}{M-b}{M-f}
\threecol{l{\'i}nea}{C-p}{C-n}
\threecol{ir al comienzo de la l{\'i}nea (o hacia al final)}{C-a}{C-e}
\threecol{oraci{\'o}n}{M-a}{M-e}
\threecol{p{\'a}rrafo}{M-\{}{M-\}}
\threecol{p{\'a}gina}{C-x [}{C-x ]}
\threecol{expresi{\'o}n s (sexp)}{C-M-b}{C-M-f}
\threecol{funci{\'o}n}{C-M-a}{C-M-e}
\threecol{ir al comienzo del buffer (o hacia al final)}{M-<}{M->}
}

\key{desplazarse hacia la siguiente pantalla}{C-v}
\key{desplazarse hacia la anterior pantalla}{M-v}
\key{desplazarse hacia la izquierda}{C-x <}
\key{desplazarse hacia la derecha}{C-x >}
\key{desplazar l{\'i}nea actual hacia el centro, arriba o abajo}{C-l}

\key{ir hacia una l{\'i}nea}{M-g g}
\key{ir hacia un car{\'a}cter}{M-g c}
\key{volver a la sangr{\'i}a}{M-m}

\section{Cortar y Eliminar}

\paralign to \hsize{#\tabskip=10pt plus 1 fil&#\tabskip=0pt&#\cr
\threecol{{\bf entidad a cortar}}{{\bf hacia atr{\'a}s}}{{\bf hacia adelante}}
\threecol{car{\'a}cter (eliminar, no cortar)}{DEL}{C-d}
\threecol{palabra}{M-DEL}{M-d}
\threecol{l{\'i}nea (hasta el final)}{M-0 C-k}{C-k}
\threecol{sentencia}{C-x DEL}{M-k}
\threecol{expresi{\'o}n s (sexp)}{M-- C-M-k}{C-M-k}
}

\key{cortar {\bf regi{\'o}n}}{C-w}
\key{copiar regi{\'o}n}{M-w}
\key{cortar a la siguiente ocurrencia del {\it car{\'a}cter}}{M-z {\it car{\'a}cter}}

\key{pegar de nuevo lo {\'u}ltimo que se cort{\'o}}{C-y}
\key{remplazar pegado reci{\'e}n con lo {\'u}ltimo que se cort{\'o}}{M-y}

\section{Selecci{\'o}n}

\key{poner selecci{\'o}n aqu{\'i}}{C-@ {\rm o} C-SPC}
\key{intercambiar la posici{\'o}n del cursor y de la selecci{\'o}n}{C-x C-x}

\key{poner selecci{\'o}n {\it arg\/} {\bf palabras} delante}{M-@}
\key{seleccionar {\bf p{\'a}rrafo}}{M-h}
\key{seleccionar {\bf p{\'a}gina}}{C-x C-p}
\key{seleccionar {\bf sexp}}{C-M-@}
\key{seleccionar {\bf funci{\'o}n}}{C-M-h}
\key{seleccionar todo el {\bf buffer}}{C-x h}

\section{Buscar y Reemplazar}

\key{remplazar interactivamente un texto (string)}{M-\%}
% query-replace-regexp is bound to C-M-% but that can't be typed on
% consoles.
\metax{usando expresiones regulares}{M-x query-replace-regexp}

Las respuestas v{\'a}lidas en el modo query-replace son:

\key{{\bf remplazar} este, ve hacia el siguiente}{SPC {\rm o} y}
\key{remplazar este, no te muevas}{,}
\key{{\bf saltar} al siguiente sin remplazar}{DEL {\rm or} n}
\key{remplazar todas las coincidencias restantes}{!}
\key{{\bf volver} a la coincidencia anterior}{^}
\key{{\bf salir} de query-replace}{RET}
\key{editar recursivamente(\kbd{C-M-c} para salir)}{C-r}

\newcolumn
\section{M{\'u}ltiple Ventanas}

Cuando dos comandos son mostrados, el segundo es un comando similar para
un frame que para una ventana.

{\setbox0=\hbox{\kbd{0}}\advance\hsize by 0\wd0
\paralign to \hsize{#\tabskip=10pt plus 1 fil&#\tabskip=0pt&#\cr
\threecol{eliminar todas las otras ventanas}{C-x 1\ \ \ \ }{C-x 5 1}
\threecol{dividir ventana, por debajo y encima}{C-x 2\ \ \ \ }{C-x 5 2}
\threecol{eliminar esta ventana}{C-x 0\ \ \ \ }{C-x 5 0}
}}
\key{dividir ventana a la mitad}{C-x 3}

\key{desplazar otra ventana}{C-M-v}

{\setbox0=\hbox{\kbd{0}}\advance\hsize by 0\wd0
\paralign to \hsize{#\tabskip=10pt plus 1 fil&#\tabskip=0pt&#\cr
\threecol{cambiar cursor a otra ventana}{C-x o}{C-x 5 o}

\threecol{seleccionar buffer en otra ventana}{C-x 4 b}{C-x 5 b}
\threecol{mostrar buffer en otra ventana}{C-x 4 C-o}{C-x 5 C-o}
\threecol{buscar archivo en otra ventana}{C-x 4 f}{C-x 5 f}
\threecol{buscar archivo de solo-lectura}{C-x 4 r}{C-x 5 r}
\threecol{en otra ventana}{}{}
\threecol{abrir Dired en otra ventana}{C-x 4 d}{C-x 5 d}
\threecol{buscar etiqueta en otra ventana}{C-x 4 .}{C-x 5 .}
}}

\key{hacer la ventana m{\'a}s alta}{C-x ^}
\key{hacer la ventana m{\'a}s estrecha}{C-x \{}
\key{hacer la ventana m{\'a}s ancha}{C-x \}}

\section{Formato y Sangr{\'i}as}

\key{sangrar {\bf l{\'i}nea} actual (dependiente del modo)}{TAB}
\key{sangrar {\bf regi{\'o}n} (dependiente del modo)}{C-M-\\}
\key{sangrar {\bf sexp} (dependiente del modo)}{C-M-q}
\key{sangrar regi{\'o}n r{\'i}gidamente en {\it arg\/} columnas}{C-x TAB}
\key{sangrar para comentar}{M-;}

\key{insertar una nueva l{\'i}nea delante del cursor}{C-o}
\key{mover resto de la l{\'i}nea verticalmente hacia abajo}{C-M-o}
\key{eliminar l{\'i}neas en blanco alrededor del cursor}{C-x C-o}
\key{unir l{\'i}nea con la anterior (con arg, next)}{M-^}
\key{eliminar todos los espacios en blanco alrededor del cursor}{M-\\}
\key{poner exactamente un espacio en el cursor}{M-SPC}

\key{rellenar p{\'a}rrafo}{M-q}
\key{establecer la columna de relleno en {\it arg}}{C-x f}
\key{establecer el prefijo con el que comienza cada l{\'i}nea}{C-x .}

\section{Cambio de Case}

\key{poner la palabra en may{\'u}sculas}{M-u}
\key{poner la palabra en min{\'u}sculas}{M-l}
\key{poner la palabra con may{\'u}scula inicial}{M-c}

\key{poner la regi{\'o}n en may{\'u}sculas}{C-x C-u}
\key{poner la regi{\'o}n en min{\'u}sculas}{C-x C-l}

\newcolumn
\title{GNU Emacs: Tarjeta de referencia}

\section{El Minibuffer}

Las siguientes teclas est{\'a}n definidas en el minibuffer.

\key{completar tanto sea posible}{TAB}
\key{completar hasta una palabra}{SPC}
\key{completar y ejecutar}{RET}
\key{mostrar posibles candidatos}{?}
\key{recuperar la entrada anterior del minibuffer}{M-p}
\key{recuperar la entrada posterior del minibuffer o la predeterminada}{M-n}
\key{buscar hacia atr{\'a}s con regexp a trav{\'e}s del historial}{M-r}
\key{buscar hacia adelante con regexp a trav{\'e}s del historial}{M-s}
\key{abortar comando}{C-g}

Presiona \kbd{C-x ESC ESC} para editar y repetir el {\'u}ltimo comando que us{\'o}
el minibuffer.  Presiona \kbd{F10} para activar los elementos de la barra del
men{\'u} en terminales de texto.

\section{buffers}

\key{seleccionar otro buffer}{C-x b}
\key{listar todos los buffers}{C-x C-b}
\key{eliminar un buffer}{C-x k}

\section{Transponiendo}

\key{transponer {\bf caracteres}}{C-t}
\key{transponer {\bf palabras}}{M-t}
\key{transponer {\bf l{\'i}neas}}{C-x C-t}
\key{transponer {\bf sexps}}{C-M-t}

\section{Revisi{\'o}n ortogr{\'a}fica}

\key{comprobar ortograf{\'i}a de la presente palabra}{M-\$}
\metax{comprobar ortograf{\'i}a de todas}{M-x ispell-region}
\metax{las palabras en la regi{\'o}n}{}
\metax{comprobar la ortograf{\'i}a de todo el buffer}{M-x ispell-buffer}
\metax{alternar la revisi{\'o}n ortogr{\'a}fica}{M-x flyspell-mode}

\section{Etiquetas (Tags)}

\key{buscar una etiqueta (una definici{\'o}n)}{M-.}
\metax{especificar un nuevo archivo}{M-x visit-tags-table}
\metax{de etiquetas}{}

\metax{buscar con regexp en todos los}{M-x tags-search}
\metax{archivos de la tabla de etiquetas}{}
\metax{ejecutar query-replace en}{M-x tags-query-replace}
\metax{todos los archivos}{}

\shortcopyrightnotice

\section{Shells}

\key{ejecutar un comando de shell}{M-!}
\key{ejecutar un comando de shell}{M-\&}
\key{asincr{\'o}nicamente}{}
\key{ejecutar un comando de shell en la regi{\'o}n}{M-|}
\key{filtrar regi{\'o}n a trav{\'e}s de un comando de shell}{C-u M-|}
\key{iniciar un shell en una ventana \kbd{*shell*}}{M-x shell}

\section{Rect{\'a}ngulos}

\key{copiar rect{\'a}ngulo al registro}{C-x r r}
\key{cortar rect{\'a}ngulo}{C-x r k}
\key{pegar rect{\'a}ngulo}{C-x r y}
\key{abrir rect{\'a}ngulo, desplazando el texto hacia la derecha}{C-x r o}
\key{poner rect{\'a}ngulo en blanco}{C-x r c}
\key{anteponer una cadena (string) a cada l{\'i}nea}{C-x r t}

\section{Abreviaturas (Abbrevs)}

\key{a{\~n}adir abreviatura global}{C-x a g}
\key{a{\~n}adir abreviatura local del modo}{C-x a l}
\key{a{\~n}adir expansi{\'o}n global}{C-x a i g}
\key{para esta abreviatura}{}
\key{a{\~n}adir expansi{\'o}n local del modo para est{\'a} abreviatura}{C-x a i l}
\key{expl{\'i}citamente expandir abreviatura}{C-x a e}

\key{expandir palabra anterior din{\'a}micamente}{M-/}

\section{Miscel{\'a}neos}

\key{inserci{\'o}n num{\'e}rica}{C-u {\it num}}
\key{inserci{\'o}n negativa}{M--}
\key{inserci{\'o}n citada}{C-q {\it car}}

\section{Expresiones regulares}

\key{cualquier {\'u}nico c{\'a}racter excepto una nueva l{\'i}nea}{. {\rm(dot)}}
\key{ninguna o m{\'a}s repeticiones}{*}
\key{una o m{\'a}s repeticiones}{+}
\key{ninguna o una repetici{\'o}n}{?}
\key{proteger car{\'a}cteres especiales}{\\}
\key{proteger car{\'a}cter especial de expresi{\'o}n regular {\it c\/}}{\\{\it c}}
\key{alternativos (``or'')}{\\|}
\key{agrupar}{\\( {\rm$\ldots$} \\)}
\key{agrupaci{\'o}n t{\'i}mida}{\\(?: {\rm$\ldots$} \\)}
\key{agrupaci{\'o}n numerada expl{\'i}cita}{\\(?NUM: {\rm$\ldots$} \\)}
\key{mismo texto del grupo n{\'u}mero {\it n\/}}{\\{\it n}}
\key{en el salto de palabra}{\\b}
\key{no en el salto de palabra}{\\B}

\paralign to \hsize{#\tabskip=10pt plus 1 fil&#\tabskip=0pt&#\cr
\threecol{{\bf entidad}}{{\bf comienzo}}{{\bf final}}
\threecol{l{\'i}nea}{^}{\$}
\threecol{palabra}{\\<}{\\>}
\threecol{s{\'i}mbolo}{\\_<}{\\_>}
\threecol{buffer}{\\`}{\\'}
%% FIXME: "`" and "'" isn't displayed correctly in the output PDF file

\threecol{{\bf clase del c{\'a}racter}}{{\bf {\'e}stas}}{{\bf otras}}
\threecol{establecer expl{\'i}citamente}{[ {\rm$\ldots$} ]}{[^ {\rm$\ldots$} ]}
\threecol{car{\'a}cter de sintaxis de palabra}{\\w}{\\W}
\threecol{car{\'a}cter con sintaxis {\it c}}{\\s{\it c}}{\\S{\it c}}
\threecol{car{\'a}cter con categor{\'i}a {\it c}}{\\c{\it c}}{\\C{\it c}}
}

\section{Caracteres internacionales}

\key{especificar idioma principal}{C-x RET l}
\metax{mostrar todos los m{\'e}todos}{M-x list-input-methods}
\key{de entrada}{}
\key{activar o desactivar el m{\'e}todo de entrada}{C-\\}
\key{establecer el sistema de codificaci{\'o}n para el siguiente comando}{C-x RET c}
\metax{mostrar todos los sistemas}{M-x list-coding-systems}
\key{de codificaci{\'o}n}{}
\metax{elegir el sistema de codificaci{\'o}n preferido}{M-x prefer-coding-system}

\newcolumn
\title{GNU Emacs: Tarjeta de referencia}
\section{Info}

\key{entrar al lector de documentaci{\'o}n de Info}{C-h i}
\key{buscar funci{\'o}n o variable especificada en Info}{C-h S}
\beginindentedkeys

Moverse dentro de un nodo:

\key{desplazarse hacia adelante}{SPC}
\key{desplazarse en reversa}{DEL}
\key{comienzo del nodo}{b}

Moverse entre nodos:

\key{{\bf siguiente} nodo}{n}
\key{nodo {\bf anterior}}{p}
\key{moverse por {\bf arriba}}{u}
\key{seleccionar un elemento del men{\'u}}{m}
\key{por el nombre}{}
\key{seleccionar el elemento {\it n\/} del men{\'u}}{{\it n}}
\key{por n{\'u}mero (1--9)}{}
\key{seguir referencia cruzada  (regresar con \kbd{l})}{f}
\key{regresar al {\'u}ltimo nodo que viste}{l}
\key{volver al nodo de directorio}{d}
\key{ir al nodo superior del archivo Info}{t}
\key{ir a cualquier nodo por nombre}{g}

Otros:

\key{ir al {\bf tutorial} de Info}{h}
\key{buscar un tema en los {\'i}ndices}{i}
\key{buscar nodos para expresiones regulares}{s}
\key{{\bf salir} de Info}{q}
\endindentedkeys

\section{Registros}

\key{guardar regi{\'o}n  en el registro}{C-x r s}
\key{insertar contenidos del registro al buffer}{C-x r i}

\key{guardar valor del cursor en el registro}{C-x r SPC}
\key{saltar al punto guardado en el registro}{C-x r j}

\section{Macros del teclado}

\key{{\bf comenzar} a definir una macro}{C-x (}
\key{{\bf terminar} la definici{\'o}n de la macro}{C-x )}
\key{{\bf ejecutar} {\'u}ltima macro definida}{C-x e}
\key{a{\~n}adir a la {\'u}ltima macro definida}{C-u C-x (}
\metax{nombrar {\'u}ltima macro definida}{M-x name-last-kbd-macro}
\metax{insertar macro como c{\'o}digo Lisp}{M-x insert-kbd-macro}
\key{en el buffer}{}

\copyrightnotice
\newcolumn
\section{Comandos para Emacs Lisp}

\key{evaluar {\bf sexp} antes del cursor}{C-x C-e}
\key{evaluar funci{\'o}n ({\bf defun}) actual}{C-M-x}
\metax{evaluar toda la {\bf regi{\'o}n}}{M-x eval-region}
\key{leer y evaluar en minibuffer}{M-:}
\metax{cargar una librer{\'i}a Lisp}{M-x load-library}
\key{desde {\bf load-path}}{}

\section{Personalizaci{\'o}n sencilla}

\metax{personalizar variables y caras}{M-x customize}

% The intended audience here is the person who wants to make simple
% customizations and knows Lisp syntax.

Creaci{\'o}n de combinaciones de teclas globales en Emacs Lisp (ejemplos):

\beginexample%
(global-set-key (kbd "C-c g") 'search-forward)
(global-set-key (kbd "M-\#") 'query-replace-regexp)
\endexample

\section{Escribiendo Comandos}

\beginexample%
(defun \<nombre-del-comando> (\<args>)
  "\<documentaci{\'o}n>"
  (interactive "\<plantilla>")
  \<cuerpo-de-la-funci{\'o}n>)
\endexample

Un ejemplo:

\beginexample%
(defun mover-l{\'i}nea-a-la-parte-superior (l{\'i}nea)
  "Reubicar l{\'i}nea a la parte superior de la ventana.
Con el argumento de prefijo L{\'I}NEA,
colocar el cursor en L{\'I}NEA."
  (interactive "P")
  (recenter (if (null l{\'i}nea)
                0
              (prefix-numeric-value l{\'i}nea))))
\endexample

La especificaci{\'o}n \kbd{interactive} dice como leer argumentos de
forma interactiva.
Presiona \kbd{C-h f interactive RET} para m{\'a}s detalles.

\bye

% Local variables:
% compile-command: "pdftex refcard"
% End:
